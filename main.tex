\documentclass[12pt, a4paper]{article} % Clase de documento: 'article' es estándar para trabajos cortos. 12pt es el tamaño de letra. A4 es el formato de papel.

% --- PAQUETES ESENCIALES ---
\usepackage[utf8]{inputenc} % Para la codificación de caracteres, esencial para acentos y eñes
\usepackage[T1]{fontenc}    % Fuente moderna para PDFs
\usepackage[spanish]{babel} % Soporte para idioma español (guionización, fechas, etc.)
\usepackage{amsmath}        % Para matemáticas avanzadas (si fuera necesario)
\usepackage{graphicx}       % Para insertar imágenes
\usepackage{booktabs}       % Mejora el aspecto de las tablas (líneas más profesionales)
\usepackage{geometry}       % Para configurar los márgenes
\usepackage{hyperref}       % Para crear enlaces en el índice y referencias (¡muy recomendado!)
\usepackage{fancyhdr}       % Para personalizar encabezados y pies de página (opcional)
\usepackage[spanish]{babel}
% ¡Asegúrate de que xcolor esté aquí!
\usepackage[table]{xcolor} % 'table' es útil para compatibilidad con listings/tablas
\usepackage{listings}       % Paquete para el formato de código fuente
% --- MODIFICACIÓN DE LA ETIQUETA 'LISTING' A 'LISTADO' ---
\addto{\captionsspanish}{\renewcommand{\lstlistingname}{Listado}}
\addto{\captionsspanish}{\renewcommand{\lstlistlistingname}{Listado de Códigos}}
% --- DEFINICIÓN DE COLOR GRIS Y ESTILO DE CÓDIGO ---

% Definimos un color gris claro para el fondo (usando el modelo RGB)
% Los valores entre 0 y 1. (0.95, 0.95, 0.95 es un gris muy claro)
\definecolor{GrisClaro}{rgb}{0.95, 0.95, 0.95}

% Configuramos el estilo por defecto para todos los bloques de código (lstlisting)
\lstset{
    language=TeX, % Lenguaje por defecto (puedes cambiarlo, ej: Python, C, Java)
    backgroundcolor=\color{GrisClaro}, % Fondo gris claro
    basicstyle=\ttfamily\small,      % Tipo de fuente: typewriter (\ttfamily) y pequeño (\small)
    breaklines=true,                 % Permite saltos de línea largos
    showstringspaces=false,          % No mostrar espacios como caracteres
    numbers=left,                    % Numeración de líneas a la izquierda
    numberstyle=\tiny\color{gray},   % Estilo del número de línea
    frame=single,                    % Dibuja un marco alrededor del código
    framesep=5pt,                    % Espacio entre el marco y el código
    rulesepcolor=\color{black},      % Color de las líneas del marco
    keywordstyle=\color{blue}\bfseries, % Palabras clave en azul y negrita
    commentstyle=\color{green!60!black}\itshape, % Comentarios en verde oscuro y cursiva
    stringstyle=\color{orange},      % Cadenas de texto en naranja
}
% --- CONFIGURACIÓN DE MARGENES ---
\geometry{
 a4paper,
 total={170mm,257mm},
 left=25mm,
 right=25mm,
 top=30mm,
 bottom=30mm,
 }

% --- CONFIGURACIÓN DE ENCABEZADOS Y PIES (OPCIONAL) ---
\pagestyle{fancy}
\fancyhf{} % Limpiar encabezados y pies
\renewcommand{\headrulewidth}{0pt} % Eliminar línea en el encabezado
\rfoot{\thepage} % Número de página a la derecha en el pie
\lfoot{ Seguridad de la Información / Leonardo Favio Najera Morales} % Texto en el pie de página izquierdo

% --- DATOS DEL DOCUMENTO (PARA LA PORTADA) ---
\newcommand{\nombreuniversidad}{Universidad Politécnica de Chiapas}
\newcommand{\nombrecarrera}{Ingeniería en Software}
\newcommand{\nombremateria}{Seguridad de la Información}
\newcommand{\nombreprofesor}{MC. José Alonso Macías Montoya}
\newcommand{\corte}{Corte/Unidad [Número de Corte]}
\newcommand{\actividad}{Actividad [Número] - [Nombre de la Actividad]}
\newcommand{\nombres}{Leonardo Favio Najera Morales}
\newcommand{\matricula}{231230}
\newcommand{\fechaentrega}{\today} % \today inserta la fecha actual automáticamente
\newcommand{\ciudad}{Ciudad, Estado}

\usepackage[utf8]{inputenc} % ¡ESTA LÍNEA ES FUNDAMENTAL!
\usepackage[T1]{fontenc}
\usepackage[spanish]{babel}
% --- INICIO DEL DOCUMENTO ---
\begin{document}

% --- PORTADA ---
\begin{titlepage}
    \centering
    \vspace*{1cm} % Espacio vertical
    {\Huge\bfseries \nombreuniversidad \par} % Título principal: Nombre de la Universidad
    \vspace{0.5cm}
    {\Large \nombrecarrera \par} % Nombre de la Carrera
    
    \vspace{2cm}
    \rule{\textwidth}{1.5pt} % Línea gruesa
    \vspace{0.4cm}
    {\Large\bfseries \actividad \par} % Título del Trabajo
    \vspace{0.4cm}
    \rule{\textwidth}{1.5pt} % Línea gruesa
    
    \vspace{3cm}
    
    % --- DATOS DEL ALUMNO Y CURSO (EN UNA SOLA COLUMNA) ---
    % Se eliminaron los entornos minipage y \hfill
    \begin{flushleft}
        \centering % Centra las líneas dentro del entorno flushleft
        \textbf{Materia:} \nombremateria \\
        \textbf{Profesor:} \nombreprofesor \\
        \textbf{Corte:} \corte
    \end{flushleft}

    \vspace{0.5cm} % Pequeño espacio de separación
    
    \begin{flushleft}
        \centering % Centra las líneas dentro del entorno flushleft
        \textbf{Alumno:} \nombres \\
        \textbf{Matrícula:} \matricula \\
    \end{flushleft}
    % --- FIN DE DATOS EN UNA SOLA COLUMNA ---
    
    \vfill % Empuja el contenido restante al final de la página
    
    % Fecha y Ciudad
    \vspace{1cm}
    \textbf{\ciudad, a \fechaentrega}
    
\end{titlepage}

\clearpage % Salto de página para empezar el contenido

% --- ÍNDICE/TABLA DE CONTENIDO ---
% Se recomienda usar la instrucción \tableofcontents inmediatamente después de la portada.
% LaTeX genera automáticamente el índice a partir de las secciones (\section, \subsection, etc.)
\tableofcontents
\clearpage

% --- CUERPO DEL TRABAJO ---

\section{Introducción}
\label{sec:introduccion}
La introducción es la primera sección formal de nuestro trabajo. Aquí se presenta el tema, el objetivo del estudio, la justificación de su importancia y la estructura general del documento.

\section{Recursos Utilizados}
\label{sec:recursos}
En esta sección se listan y describen brevemente las herramientas, software, bibliografía o fuentes de datos que fueron fundamentales para la realización de la actividad.

\begin{itemize}
    \item \textbf{Software:} Overleaf y compilador \LaTeX{}.
    \item \textbf{Bibliografía:} Libros y artículos relacionados con la plantilla de trabajos académicos.
    \item \textbf{Otros:} Datos de la universidad, nombre de la carrera, etc.
\end{itemize}

\section{Desarrollo}
\label{sec:desarrollo}
Esta es la parte central del trabajo. Se presenta la metodología, el marco teórico y la ejecución de los pasos necesarios para cumplir con el objetivo.

\subsection{Texto de Ejemplo y Estructura}
\label{sec:texto}
El texto en \LaTeX{} se escribe de forma natural. Los comandos como \verb|\section{}| y \verb|\subsection{}| son claves para estructurar el documento y generar automáticamente el índice. Para enfatizar, podemos usar \textbf{negritas} o \textit{cursivas}.

\subsubsection{Sub-subsección de Ejemplo}
Aquí se puede profundizar en un punto específico. Un entorno de lista numerada se crea con \texttt{enumerate}:
\begin{enumerate}
    \item Primer punto de la lista.
    \item Segundo punto importante.
\end{enumerate}

\subsection{Ejemplo de Inserción de Imágenes}
Para insertar una imagen, debes subir el archivo (por ejemplo, \texttt{logouniversidad.png}) a tu proyecto de Overleaf. Usamos el entorno \texttt{figure} para que la imagen se coloque correctamente y pueda tener un título (caption) y una referencia (label).

\begin{figure}[h] % [h] intenta colocar la figura "aquí" (here)
    \centering
    % \includegraphics[width=0.8\textwidth]{logouniversidad.png} % Descomentar y cambiar "logouniversidad.png"
    \includegraphics[width=0.7\textwidth]{tecnologias.jpg} 
    \caption{Ejemplo de una imagen o gráfico insertado en el documento.}
    \label{fig:ejemploimagen}
\end{figure}

Podemos referenciar la figura automáticamente: Ver la Figura \ref{fig:ejemploimagen}.

\subsection{Ejemplo de Inserción de Tablas}
Para las tablas, el entorno \texttt{tabular} es el más común, pero \texttt{booktabs} mejora mucho su aspecto. Usamos \texttt{table} para que la tabla sea flotante y tenga título y referencia.

\begin{table}[h]
    \centering
    \caption{Resultados Obtenidos en la Práctica}
    \label{tab:ejemplotabla}
    \begin{tabular}{lccc} % l: izquierda, c: centrado, r: derecha
        \toprule % Línea superior de booktabs
        \textbf{Variable} & \textbf{Valor Inicial} & \textbf{Valor Final} & \textbf{Diferencia} \\
        \midrule % Línea intermedia de booktabs
        Temperatura ($^{\circ}$C) & 25.0 & 32.5 & +7.5 \\
        Presión (kPa) & 101.3 & 100.9 & -0.4 \\
        Humedad relativa (\%) & 65 & 58 & -7 \\
        \bottomrule % Línea inferior de booktabs
    \end{tabular}
    \vspace{0.5em} % Espacio vertical
    \footnotesize Nota: Los datos son meramente ilustrativos para el ejemplo.
\end{table}

Podemos referenciar la tabla: Consultar los datos en la Tabla \ref{tab:ejemplotabla}.

\section{Resultados y Conclusiones}
\label{sec:resultados}
En esta sección se presentan los \textbf{resultados clave} derivados del desarrollo, seguidos por las \textbf{conclusiones}. La conclusión debe ser un resumen conciso de los hallazgos y si se lograron los objetivos planteados en la introducción.
\begin{itemize}
    \item \textbf{Resultado 1:} Se logró definir una plantilla funcional.
    \item \textbf{Conclusión:} \LaTeX{} es una herramienta poderosa para la redacción académica.
\end{itemize}

\clearpage % Salto de página para empezar los anexos

\section{Anexos}
\label{sec:anexos}
Los anexos incluyen material complementario que no es esencial para la comprensión del cuerpo principal del trabajo, pero que lo respalda (ej. capturas de pantalla, código fuente extenso, encuestas, etc.).

\subsection{Capturas de pantalla Adicional}
Se puede incluir acciones ó código en un entorno especial (si usas el paquete \texttt{listings} o similar).

\begin{figure}[h] % [h] intenta colocar la figura "aquí" (here)
    \centering
    % \includegraphics[width=0.8\textwidth]{logouniversidad.png} % Descomentar y cambiar "logouniversidad.png"
    \includegraphics[width=0.7\textwidth]{tecnologias2.jpg} 
    \caption{Segundo Ejemplo de una imagen o gráfico insertado en el documento.}
    \label{fig:ejemploimagen}
\end{figure}

\begin{verbatim}
% Ejemplo de código o texto sin formato
if (resultado > 100) {
    print("Éxito");
} else {
    print("Fallo");
}
\end{verbatim}

\clearpage % Salto de página para empezar los anexos

\subsection{Código Fuente Adicional (Con Estilo Mejorado)}
Para incluir código fuente de manera profesional y con estilo, utilizamos el entorno \texttt{lstlisting}. Si necesitas cambiar el lenguaje para resaltar la sintaxis correctamente, simplemente añade la opción \texttt{language}.

Texto antes del espacio.
\vspace{1cm} % Inserta un espacio vertical de 1 centímetro de altura.
Texto después del espacio de 1 cm.

\vspace{5ex} % Inserta un espacio vertical equivalente a 5 veces la altura de la 'x' de la fuente actual.
Otro texto.

\begin{lstlisting}[language=Python, caption={Ejemplo de un script en Python}]
# Definición de una función simple
def saludar(nombre):
    """Esta función saluda a la persona pasada como parámetro."""
    mensaje = "Hola, " + nombre + ". Bienvenido a Overleaf."
    return mensaje

# Uso de la función
usuario = "Estudiante"
resultado = saludar(usuario)
print(resultado)
\end{lstlisting}

\end{document}