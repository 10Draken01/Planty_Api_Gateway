\documentclass[12pt,a4paper]{article}
\usepackage[utf8]{inputenc}
\usepackage[spanish]{babel}
\usepackage{geometry}
\usepackage{graphicx}
\usepackage{booktabs}
\usepackage{longtable}
\usepackage{xcolor}
\usepackage{hyperref}
\usepackage{fancyhdr}
\usepackage{array}
\usepackage{multirow}

\geometry{margin=2.5cm}
\pagestyle{fancy}
\fancyhf{}
\fancyhead[L]{\small Medición COSMIC - ApiGateway}
\fancyhead[R]{\small \today}
\fancyfoot[C]{\thepage}

\hypersetup{
    colorlinks=true,
    linkcolor=blue,
    citecolor=blue,
    urlcolor=blue
}

\title{\textbf{MEDICIÓN DE PUNTOS FUNCIÓN COSMIC}\\
\large Sistema ApiGateway - Arquitectura de Microservicios}
\author{Especialista Certificado COSMIC (CCFL)}
\date{\today}

\begin{document}

\maketitle
\thispagestyle{empty}
\newpage

\tableofcontents
\newpage

\section{Resumen Ejecutivo}

El presente documento reporta la medición funcional del proyecto \textbf{ApiGateway}, un sistema distribuido basado en arquitectura de microservicios, utilizando el método COSMIC (Common Software Measurement International Consortium) versión 4.0.

\subsection{Resultado de la Medición}

\begin{center}
\begin{tabular}{|l|r|}
\hline
\textbf{Concepto} & \textbf{Valor} \\
\hline
\hline
Total de Procesos Funcionales Identificados & 23 \\
\hline
\textbf{Total de Puntos Función COSMIC (CFPs)} & \textbf{117} \\
\hline
\hline
\multicolumn{2}{|l|}{\textit{Desagregado por tipo de movimiento:}} \\
\hline
Movimientos Entry (E) & 51 \\
Movimientos Exit (X) & 23 \\
Movimientos Read (R) & 25 \\
Movimientos Write (W) & 18 \\
\hline
\end{tabular}
\end{center}

\subsection{Distribución por Microservicio}

\begin{center}
\begin{tabular}{|l|c|r|}
\hline
\textbf{Microservicio} & \textbf{Procesos} & \textbf{CFPs} \\
\hline
Authentication Service & 3 & 12 \\
Users Service & 5 & 29 \\
Chatbot Service (RAG) & 6 & 32 \\
Orchard Service & 9 & 44 \\
\hline
\textbf{Total} & \textbf{23} & \textbf{117} \\
\hline
\end{tabular}
\end{center}

\subsection{Interpretación}

El sistema presenta un tamaño funcional de \textbf{117 CFPs}, lo cual corresponde a un sistema de mediana complejidad con funcionalidades distribuidas en cuatro microservicios independientes. El servicio de Orchard representa el 37.6\% del tamaño funcional total, seguido por el servicio de Chatbot con RAG (27.4\%), evidenciando la complejidad de las operaciones de gestión agrícola y procesamiento de lenguaje natural con recuperación aumentada.

\newpage

\section{Alcance del Conteo}

\subsection{Propósito de la Medición}

Esta medición COSMIC tiene como propósitos:

\begin{itemize}
    \item Cuantificar el tamaño funcional del sistema para estimación de esfuerzo y costos
    \item Proporcionar una métrica objetiva para planificación de proyecto
    \item Establecer una línea base para mediciones futuras y control de cambios
    \item Facilitar la comunicación técnica con stakeholders mediante métricas estándar
    \item Permitir benchmarking con proyectos similares en la industria
\end{itemize}

\subsection{Alcance Funcional}

El alcance de la medición incluye todos los procesos funcionales implementados en los siguientes componentes:

\begin{enumerate}
    \item \textbf{API Gateway}: Componente de enrutamiento y proxy (sin contabilización de procesos propios según COSMIC 4.0)
    \item \textbf{Authentication Service}: Gestión de autenticación JWT y validación de tokens
    \item \textbf{Users Service}: Operaciones CRUD completas sobre entidades de usuario
    \item \textbf{Chatbot Service}: Inteligencia artificial conversacional con RAG (Retrieval-Augmented Generation) incluyendo:
    \begin{itemize}
        \item Gestión de documentos PDF
        \item Procesamiento de embeddings vectoriales
        \item Búsqueda semántica
        \item Generación de respuestas con LLM
    \end{itemize}
    \item \textbf{Orchard Service}: Gestión completa de huertos agrícolas y sus plantas asociadas
\end{enumerate}

\subsection{Exclusiones}

\textbf{No se incluyen en esta medición:}

\begin{itemize}
    \item Servicios de notificaciones (mencionados en router pero sin implementación visible)
    \item Servicios de algoritmo genético (mencionados en router pero sin implementación visible)
    \item Funciones de infraestructura pura (logging, monitoring, health checks simples)
    \item Servicios externos (Ollama LLM, MongoDB, ChromaDB) considerados fuera del límite
\end{itemize}

\newpage

\section{Metodología COSMIC Aplicada}

\subsection{Marco Conceptual}

La medición COSMIC (versión 4.0) se basa en los siguientes principios fundamentales:

\subsubsection{Límite del Software}

El límite del software define la frontera conceptual entre:
\begin{itemize}
    \item \textbf{Interior}: El software medido (los 4 microservicios)
    \item \textbf{Exterior}: Usuarios funcionales y almacenamiento persistente
\end{itemize}

Para este proyecto:
\begin{itemize}
    \item \textbf{Usuarios Funcionales}: Aplicación cliente, administrador, servicio Ollama (LLM externo)
    \item \textbf{Almacenamiento Persistente}: MongoDB, ChromaDB, sistema de archivos
\end{itemize}

\subsubsection{Proceso Funcional}

Un proceso funcional es un conjunto cohesivo de movimientos de datos que implementa un requisito funcional único. Cada endpoint REST que manipula datos constituye típicamente un proceso funcional independiente.

\subsubsection{Movimientos de Datos}

COSMIC reconoce cuatro tipos de movimientos de datos:

\begin{enumerate}
    \item \textbf{Entry (E)}: Movimiento de datos desde un usuario funcional hacia el proceso
    \item \textbf{Exit (X)}: Movimiento de datos desde el proceso hacia un usuario funcional
    \item \textbf{Read (R)}: Movimiento de datos desde almacenamiento persistente al proceso
    \item \textbf{Write (W)}: Movimiento de datos desde el proceso a almacenamiento persistente
\end{enumerate}

\textbf{Regla fundamental}: 1 movimiento de datos = 1 CFP

\subsection{Proceso de Medición Aplicado}

\begin{enumerate}
    \item \textbf{Identificación del límite}: Análisis de arquitectura de microservicios
    \item \textbf{Identificación de procesos funcionales}: Mapeo de endpoints REST a requisitos funcionales
    \item \textbf{Identificación de movimientos}: Análisis de código fuente (controladores, casos de uso, repositorios)
    \item \textbf{Aplicación de reglas de medición}: Contabilización según estándar COSMIC 4.0
    \item \textbf{Agregación}: Suma de movimientos por proceso, servicio y total
\end{enumerate}

\newpage

\section{Procesos Funcionales Identificados}

Esta sección detalla todos los procesos funcionales medidos, organizados por microservicio.

\subsection{Authentication Service (12 CFPs)}

\subsubsection{PF-AUTH-001: Registro de Usuario (6 CFPs)}

\textbf{Descripción}: Permite a un nuevo usuario registrarse en el sistema proporcionando credenciales.

\textbf{Endpoint}: \texttt{POST /api/auth/register}

\textbf{Movimientos de Datos}:
\begin{center}
\begin{tabular}{|l|c|l|}
\hline
\textbf{Tipo} & \textbf{Cantidad} & \textbf{Descripción} \\
\hline
Entry (E) & 3 & name, email, password \\
Read (R) & 1 & Consulta de usuario existente en Users Service \\
Write (W) & 1 & Creación de usuario en Users Service \\
Exit (X) & 1 & Token JWT + datos de usuario \\
\hline
\textbf{Total CFPs} & \textbf{6} & \\
\hline
\end{tabular}
\end{center}

\textbf{Evidencia}: \texttt{authentication/src/application/usecases/RegisterUseCase.ts:22}

\subsubsection{PF-AUTH-002: Inicio de Sesión (4 CFPs)}

\textbf{Descripción}: Autentica a un usuario existente y genera token de sesión.

\textbf{Endpoint}: \texttt{POST /api/auth/login}

\textbf{Movimientos de Datos}:
\begin{center}
\begin{tabular}{|l|c|l|}
\hline
\textbf{Tipo} & \textbf{Cantidad} & \textbf{Descripción} \\
\hline
Entry (E) & 2 & email, password \\
Read (R) & 1 & Consulta de usuario por email en Users Service \\
Exit (X) & 1 & Token JWT + datos de usuario \\
\hline
\textbf{Total CFPs} & \textbf{4} & \\
\hline
\end{tabular}
\end{center}

\textbf{Evidencia}: \texttt{authentication/src/application/usecases/LoginUseCase.ts}

\subsubsection{PF-AUTH-003: Validación de Token (2 CFPs)}

\textbf{Descripción}: Valida un token JWT y retorna información del usuario autenticado.

\textbf{Endpoint}: \texttt{POST /api/auth/validate}

\textbf{Movimientos de Datos}:
\begin{center}
\begin{tabular}{|l|c|l|}
\hline
\textbf{Tipo} & \textbf{Cantidad} & \textbf{Descripción} \\
\hline
Entry (E) & 1 & token \\
Exit (X) & 1 & resultado de validación + userId + email \\
\hline
\textbf{Total CFPs} & \textbf{2} & \\
\hline
\end{tabular}
\end{center}

\textbf{Evidencia}: \texttt{authentication/src/application/usecases/ValidateTokenUseCase.ts}

\subsection{Users Service (29 CFPs)}

\subsubsection{PF-USER-001: Creación de Usuario (9 CFPs)}

\textbf{Descripción}: Crea un nuevo usuario en la base de datos con todos sus atributos.

\textbf{Endpoint}: \texttt{POST /api/users/create}

\textbf{Movimientos de Datos}:
\begin{center}
\begin{tabular}{|l|c|l|}
\hline
\textbf{Tipo} & \textbf{Cantidad} & \textbf{Descripción} \\
\hline
Entry (E) & 6 & name, email, password, experience\_level, \\
& & profile\_image, historyTimeUse\_ids \\
Read (R) & 1 & Verificación de usuario existente en MongoDB \\
Write (W) & 1 & Persistencia de usuario en MongoDB \\
Exit (X) & 1 & Datos completos del usuario creado \\
\hline
\textbf{Total CFPs} & \textbf{9} & \\
\hline
\end{tabular}
\end{center}

\textbf{Evidencia}: \texttt{api-users/src/presentation/controllers/UserController.ts:18}

\subsubsection{PF-USER-002: Consulta de Usuario por ID (3 CFPs)}

\textbf{Descripción}: Obtiene los datos de un usuario específico por su identificador.

\textbf{Endpoint}: \texttt{GET /api/users/:id}

\textbf{Movimientos de Datos}:
\begin{center}
\begin{tabular}{|l|c|l|}
\hline
\textbf{Tipo} & \textbf{Cantidad} & \textbf{Descripción} \\
\hline
Entry (E) & 1 & id (parámetro de ruta) \\
Read (R) & 1 & Consulta de usuario en MongoDB \\
Exit (X) & 1 & Datos del usuario (sin contraseña) \\
\hline
\textbf{Total CFPs} & \textbf{3} & \\
\hline
\end{tabular}
\end{center}

\textbf{Evidencia}: \texttt{api-users/src/presentation/controllers/UserController.ts:108}

\subsubsection{PF-USER-003: Consulta de Usuario por Email (3 CFPs)}

\textbf{Descripción}: Obtiene usuario por email (incluye contraseña para autenticación).

\textbf{Endpoint}: \texttt{GET /api/users/email/:email}

\textbf{Movimientos de Datos}:
\begin{center}
\begin{tabular}{|l|c|l|}
\hline
\textbf{Tipo} & \textbf{Cantidad} & \textbf{Descripción} \\
\hline
Entry (E) & 1 & email (parámetro de ruta) \\
Read (R) & 1 & Consulta de usuario en MongoDB \\
Exit (X) & 1 & Datos completos del usuario (incluye password hash) \\
\hline
\textbf{Total CFPs} & \textbf{3} & \\
\hline
\end{tabular}
\end{center}

\textbf{Evidencia}: \texttt{api-users/src/presentation/controllers/UserController.ts:153}

\subsubsection{PF-USER-004: Actualización de Usuario (10 CFPs)}

\textbf{Descripción}: Actualiza los datos de un usuario existente.

\textbf{Endpoint}: \texttt{PUT /api/users/:id}

\textbf{Movimientos de Datos}:
\begin{center}
\begin{tabular}{|l|c|l|}
\hline
\textbf{Tipo} & \textbf{Cantidad} & \textbf{Descripción} \\
\hline
Entry (E) & 6 & id, name, email, password, experience\_level, \\
& & profile\_image (todos opcionales excepto id) \\
Read (R) & 2 & Consulta de usuario existente + \\
& & Verificación de email duplicado \\
Write (W) & 1 & Actualización de usuario en MongoDB \\
Exit (X) & 1 & Datos actualizados del usuario \\
\hline
\textbf{Total CFPs} & \textbf{10} & \\
\hline
\end{tabular}
\end{center}

\textbf{Evidencia}: \texttt{api-users/src/presentation/controllers/UserController.ts:191}

\subsubsection{PF-USER-005: Eliminación de Usuario (4 CFPs)}

\textbf{Descripción}: Elimina un usuario del sistema de forma permanente.

\textbf{Endpoint}: \texttt{DELETE /api/users/:id}

\textbf{Movimientos de Datos}:
\begin{center}
\begin{tabular}{|l|c|l|}
\hline
\textbf{Tipo} & \textbf{Cantidad} & \textbf{Descripción} \\
\hline
Entry (E) & 1 & id (parámetro de ruta) \\
Read (R) & 1 & Verificación de existencia en MongoDB \\
Write (W) & 1 & Eliminación de usuario en MongoDB \\
Exit (X) & 1 & Mensaje de confirmación \\
\hline
\textbf{Total CFPs} & \textbf{4} & \\
\hline
\end{tabular}
\end{center}

\textbf{Evidencia}: \texttt{api-users/src/presentation/controllers/UserController.ts:280}

\subsection{Chatbot Service (32 CFPs)}

\subsubsection{PF-CHAT-001: Envío de Mensaje al Chatbot (9 CFPs)}

\textbf{Descripción}: Procesa un mensaje del usuario, realiza búsqueda semántica en documentos y genera respuesta con LLM utilizando arquitectura RAG (Retrieval-Augmented Generation).

\textbf{Endpoint}: \texttt{POST /api/chat/message}

\textbf{Movimientos de Datos}:
\begin{center}
\begin{tabular}{|l|c|l|}
\hline
\textbf{Tipo} & \textbf{Cantidad} & \textbf{Descripción} \\
\hline
Entry (E) & 3 & message, includeContext, maxContextChunks \\
Read (R) & 3 & Generación de embedding (Ollama) + \\
& & Búsqueda semántica (ChromaDB) + \\
& & Consulta al LLM (Ollama) \\
Write (W) & 2 & Almacenamiento de mensaje del usuario + \\
& & Almacenamiento de respuesta \\
Exit (X) & 1 & Respuesta generada por el LLM \\
\hline
\textbf{Total CFPs} & \textbf{9} & \\
\hline
\end{tabular}
\end{center}

\textbf{Evidencia}: \texttt{api-chatbot/src/application/use-cases/SendMessageUseCase.ts}

\subsubsection{PF-CHAT-002: Consulta de Historial de Chat (3 CFPs)}

\textbf{Descripción}: Obtiene el historial completo de conversación de una sesión.

\textbf{Endpoint}: \texttt{GET /api/chat/history/:sessionId}

\textbf{Movimientos de Datos}:
\begin{center}
\begin{tabular}{|l|c|l|}
\hline
\textbf{Tipo} & \textbf{Cantidad} & \textbf{Descripción} \\
\hline
Entry (E) & 1 & sessionId (parámetro de ruta) \\
Read (R) & 1 & Consulta de mensajes en repositorio \\
Exit (X) & 1 & Lista de mensajes de la sesión \\
\hline
\textbf{Total CFPs} & \textbf{3} & \\
\hline
\end{tabular}
\end{center}

\textbf{Evidencia}: \texttt{api-chatbot/src/presentation/controllers/ChatController.ts:57}

\subsubsection{PF-DOC-001: Carga de Documento PDF (5 CFPs)}

\textbf{Descripción}: Sube un archivo PDF al sistema y registra sus metadatos.

\textbf{Endpoint}: \texttt{POST /api/documents/upload}

\textbf{Movimientos de Datos}:
\begin{center}
\begin{tabular}{|l|c|l|}
\hline
\textbf{Tipo} & \textbf{Cantidad} & \textbf{Descripción} \\
\hline
Entry (E) & 2 & file (archivo PDF multipart), metadata (JSON) \\
Write (W) & 2 & Almacenamiento del archivo + \\
& & Registro de documento en repositorio \\
Exit (X) & 1 & Información del documento creado \\
\hline
\textbf{Total CFPs} & \textbf{5} & \\
\hline
\end{tabular}
\end{center}

\textbf{Evidencia}: \texttt{api-chatbot/src/application/use-cases/UploadDocumentUseCase.ts}

\subsubsection{PF-DOC-002: Procesamiento de Documento (8 CFPs)}

\textbf{Descripción}: Extrae texto del PDF, divide en chunks, genera embeddings y almacena en ChromaDB.

\textbf{Endpoint}: \texttt{POST /api/documents/:id/process}

\textbf{Movimientos de Datos}:
\begin{center}
\begin{tabular}{|l|c|l|}
\hline
\textbf{Tipo} & \textbf{Cantidad} & \textbf{Descripción} \\
\hline
Entry (E) & 3 & id, chunkSize, chunkOverlap \\
Read (R) & 2 & Lectura del PDF + Generación de embeddings \\
Write (W) & 2 & Almacenamiento de chunks + \\
& & Almacenamiento de vectores (ChromaDB) \\
Exit (X) & 1 & Resultado del procesamiento \\
\hline
\textbf{Total CFPs} & \textbf{8} & \\
\hline
\end{tabular}
\end{center}

\textbf{Evidencia}: \texttt{api-chatbot/src/application/use-cases/ProcessDocumentUseCase.ts}

\subsubsection{PF-DOC-003: Listar Documentos (2 CFPs)}

\textbf{Descripción}: Obtiene la lista de todos los documentos cargados en el sistema.

\textbf{Endpoint}: \texttt{GET /api/documents}

\textbf{Movimientos de Datos}:
\begin{center}
\begin{tabular}{|l|c|l|}
\hline
\textbf{Tipo} & \textbf{Cantidad} & \textbf{Descripción} \\
\hline
Read (R) & 1 & Consulta de todos los documentos \\
Exit (X) & 1 & Lista de documentos con metadatos \\
\hline
\textbf{Total CFPs} & \textbf{2} & \\
\hline
\end{tabular}
\end{center}

\textbf{Evidencia}: \texttt{api-chatbot/src/application/use-cases/GetDocumentsUseCase.ts}

\subsubsection{PF-DOC-004: Eliminación de Documento (5 CFPs)}

\textbf{Descripción}: Elimina un documento y sus vectores asociados del sistema.

\textbf{Endpoint}: \texttt{DELETE /api/documents/:id}

\textbf{Movimientos de Datos}:
\begin{center}
\begin{tabular}{|l|c|l|}
\hline
\textbf{Tipo} & \textbf{Cantidad} & \textbf{Descripción} \\
\hline
Entry (E) & 1 & id (documento a eliminar) \\
Read (R) & 1 & Consulta de documento en repositorio \\
Write (W) & 2 & Eliminación del archivo + \\
& & Eliminación de vectores (ChromaDB) \\
Exit (X) & 1 & Confirmación de eliminación \\
\hline
\textbf{Total CFPs} & \textbf{5} & \\
\hline
\end{tabular}
\end{center}

\textbf{Evidencia}: \texttt{api-chatbot/src/application/use-cases/DeleteDocumentUseCase.ts}

\subsection{Orchard Service (44 CFPs)}

\subsubsection{PF-ORC-001: Creación de Huerto (7 CFPs)}

\textbf{Descripción}: Crea un nuevo huerto con sus características básicas.

\textbf{Endpoint}: \texttt{POST /api/orchards}

\textbf{Movimientos de Datos}:
\begin{center}
\begin{tabular}{|l|c|l|}
\hline
\textbf{Tipo} & \textbf{Cantidad} & \textbf{Descripción} \\
\hline
Entry (E) & 5 & name, location, size, soilType, climate \\
Write (W) & 1 & Persistencia en MongoDB \\
Exit (X) & 1 & Datos del huerto creado \\
\hline
\textbf{Total CFPs} & \textbf{7} & \\
\hline
\end{tabular}
\end{center}

\textbf{Evidencia}: \texttt{api-orchard/src/application/use-cases/CreateOrchardUseCase.ts}

\subsubsection{PF-ORC-002: Consulta de Huerto por ID (3 CFPs)}

\textbf{Descripción}: Obtiene la información detallada de un huerto específico.

\textbf{Endpoint}: \texttt{GET /api/orchards/:id}

\textbf{Movimientos de Datos}:
\begin{center}
\begin{tabular}{|l|c|l|}
\hline
\textbf{Tipo} & \textbf{Cantidad} & \textbf{Descripción} \\
\hline
Entry (E) & 1 & id (identificador del huerto) \\
Read (R) & 1 & Consulta en MongoDB \\
Exit (X) & 1 & Datos completos del huerto \\
\hline
\textbf{Total CFPs} & \textbf{3} & \\
\hline
\end{tabular}
\end{center}

\textbf{Evidencia}: \texttt{api-orchard/src/application/use-cases/GetOrchardUseCase.ts}

\subsubsection{PF-ORC-003: Listar Huertos (3 CFPs)}

\textbf{Descripción}: Obtiene lista de huertos con filtro opcional por estado activo.

\textbf{Endpoint}: \texttt{GET /api/orchards}

\textbf{Movimientos de Datos}:
\begin{center}
\begin{tabular}{|l|c|l|}
\hline
\textbf{Tipo} & \textbf{Cantidad} & \textbf{Descripción} \\
\hline
Entry (E) & 1 & active (query param: true/false/undefined) \\
Read (R) & 1 & Consulta filtrada en MongoDB \\
Exit (X) & 1 & Lista de huertos \\
\hline
\textbf{Total CFPs} & \textbf{3} & \\
\hline
\end{tabular}
\end{center}

\textbf{Evidencia}: \texttt{api-orchard/src/application/use-cases/ListOrchardsUseCase.ts}

\subsubsection{PF-ORC-004: Actualización de Huerto (9 CFPs)}

\textbf{Descripción}: Modifica los datos de un huerto existente.

\textbf{Endpoint}: \texttt{PUT /api/orchards/:id}

\textbf{Movimientos de Datos}:
\begin{center}
\begin{tabular}{|l|c|l|}
\hline
\textbf{Tipo} & \textbf{Cantidad} & \textbf{Descripción} \\
\hline
Entry (E) & 6 & id, name, location, size, soilType, climate \\
Read (R) & 1 & Consulta de huerto existente en MongoDB \\
Write (W) & 1 & Actualización en MongoDB \\
Exit (X) & 1 & Datos actualizados del huerto \\
\hline
\textbf{Total CFPs} & \textbf{9} & \\
\hline
\end{tabular}
\end{center}

\textbf{Evidencia}: \texttt{api-orchard/src/application/use-cases/UpdateOrchardUseCase.ts}

\subsubsection{PF-ORC-005: Eliminación de Huerto (4 CFPs)}

\textbf{Descripción}: Elimina un huerto del sistema permanentemente.

\textbf{Endpoint}: \texttt{DELETE /api/orchards/:id}

\textbf{Movimientos de Datos}:
\begin{center}
\begin{tabular}{|l|c|l|}
\hline
\textbf{Tipo} & \textbf{Cantidad} & \textbf{Descripción} \\
\hline
Entry (E) & 1 & id (identificador) \\
Read (R) & 1 & Verificación de existencia en MongoDB \\
Write (W) & 1 & Eliminación en MongoDB \\
Exit (X) & 1 & Confirmación de eliminación \\
\hline
\textbf{Total CFPs} & \textbf{4} & \\
\hline
\end{tabular}
\end{center}

\textbf{Evidencia}: \texttt{api-orchard/src/application/use-cases/DeleteOrchardUseCase.ts}

\subsubsection{PF-ORC-006: Activar Huerto (4 CFPs)}

\textbf{Descripción}: Cambia el estado de un huerto a activo.

\textbf{Endpoint}: \texttt{PATCH /api/orchards/:id/activate}

\textbf{Movimientos de Datos}:
\begin{center}
\begin{tabular}{|l|c|l|}
\hline
\textbf{Tipo} & \textbf{Cantidad} & \textbf{Descripción} \\
\hline
Entry (E) & 1 & id (identificador) \\
Read (R) & 1 & Consulta de huerto en MongoDB \\
Write (W) & 1 & Actualización de estado a activo \\
Exit (X) & 1 & Huerto con estado actualizado \\
\hline
\textbf{Total CFPs} & \textbf{4} & \\
\hline
\end{tabular}
\end{center}

\textbf{Evidencia}: \texttt{api-orchard/src/application/use-cases/ToggleOrchardStateUseCase.ts}

\subsubsection{PF-ORC-007: Desactivar Huerto (4 CFPs)}

\textbf{Descripción}: Cambia el estado de un huerto a inactivo.

\textbf{Endpoint}: \texttt{PATCH /api/orchards/:id/deactivate}

\textbf{Movimientos de Datos}:
\begin{center}
\begin{tabular}{|l|c|l|}
\hline
\textbf{Tipo} & \textbf{Cantidad} & \textbf{Descripción} \\
\hline
Entry (E) & 1 & id (identificador) \\
Read (R) & 1 & Consulta de huerto en MongoDB \\
Write (W) & 1 & Actualización de estado a inactivo \\
Exit (X) & 1 & Huerto con estado actualizado \\
\hline
\textbf{Total CFPs} & \textbf{4} & \\
\hline
\end{tabular}
\end{center}

\textbf{Evidencia}: \texttt{api-orchard/src/application/use-cases/ToggleOrchardStateUseCase.ts}

\subsubsection{PF-ORC-008: Agregar Planta a Huerto (5 CFPs)}

\textbf{Descripción}: Asocia una planta a un huerto existente.

\textbf{Endpoint}: \texttt{POST /api/orchards/:id/plants}

\textbf{Movimientos de Datos}:
\begin{center}
\begin{tabular}{|l|c|l|}
\hline
\textbf{Tipo} & \textbf{Cantidad} & \textbf{Descripción} \\
\hline
Entry (E) & 2 & id (huerto), plantId \\
Read (R) & 1 & Consulta de huerto en MongoDB \\
Write (W) & 1 & Actualización del array de plantas \\
Exit (X) & 1 & Huerto actualizado con nueva planta \\
\hline
\textbf{Total CFPs} & \textbf{5} & \\
\hline
\end{tabular}
\end{center}

\textbf{Evidencia}: \texttt{api-orchard/src/application/use-cases/ManagePlantsUseCase.ts}

\subsubsection{PF-ORC-009: Remover Planta de Huerto (5 CFPs)}

\textbf{Descripción}: Elimina una planta asociada de un huerto.

\textbf{Endpoint}: \texttt{DELETE /api/orchards/:id/plants/:plantId}

\textbf{Movimientos de Datos}:
\begin{center}
\begin{tabular}{|l|c|l|}
\hline
\textbf{Tipo} & \textbf{Cantidad} & \textbf{Descripción} \\
\hline
Entry (E) & 2 & id (huerto), plantId \\
Read (R) & 1 & Consulta de huerto en MongoDB \\
Write (W) & 1 & Actualización del array de plantas \\
Exit (X) & 1 & Huerto actualizado sin la planta \\
\hline
\textbf{Total CFPs} & \textbf{5} & \\
\hline
\end{tabular}
\end{center}

\textbf{Evidencia}: \texttt{api-orchard/src/application/use-cases/ManagePlantsUseCase.ts}

\newpage

\section{Tabla Consolidada de Medición}

\begin{longtable}{|l|l|c|c|c|c|c|c|}
\caption{Resumen de Procesos Funcionales y Movimientos COSMIC} \\
\hline
\textbf{ID} & \textbf{Proceso Funcional} & \textbf{E} & \textbf{X} & \textbf{R} & \textbf{W} & \textbf{CFP} & \textbf{Servicio} \\
\hline
\endfirsthead

\multicolumn{8}{c}{\textit{(Continuación de la tabla)}} \\
\hline
\textbf{ID} & \textbf{Proceso Funcional} & \textbf{E} & \textbf{X} & \textbf{R} & \textbf{W} & \textbf{CFP} & \textbf{Servicio} \\
\hline
\endhead

\hline
\multicolumn{8}{r}{\textit{(Continúa en la siguiente página)}} \\
\endfoot

\hline
\endlastfoot

\multicolumn{8}{|l|}{\cellcolor{gray!30}\textbf{AUTHENTICATION SERVICE}} \\
\hline
PF-AUTH-001 & Registro de Usuario & 3 & 1 & 1 & 1 & 6 & Auth \\
PF-AUTH-002 & Inicio de Sesión & 2 & 1 & 1 & 0 & 4 & Auth \\
PF-AUTH-003 & Validación de Token & 1 & 1 & 0 & 0 & 2 & Auth \\
\hline
\multicolumn{2}{|l|}{\textbf{Subtotal Authentication}} & 6 & 3 & 2 & 1 & \textbf{12} & \\
\hline
\hline
\multicolumn{8}{|l|}{\cellcolor{gray!30}\textbf{USERS SERVICE}} \\
\hline
PF-USER-001 & Creación de Usuario & 6 & 1 & 1 & 1 & 9 & Users \\
PF-USER-002 & Consulta por ID & 1 & 1 & 1 & 0 & 3 & Users \\
PF-USER-003 & Consulta por Email & 1 & 1 & 1 & 0 & 3 & Users \\
PF-USER-004 & Actualización de Usuario & 6 & 1 & 2 & 1 & 10 & Users \\
PF-USER-005 & Eliminación de Usuario & 1 & 1 & 1 & 1 & 4 & Users \\
\hline
\multicolumn{2}{|l|}{\textbf{Subtotal Users}} & 15 & 5 & 6 & 3 & \textbf{29} & \\
\hline
\hline
\multicolumn{8}{|l|}{\cellcolor{gray!30}\textbf{CHATBOT SERVICE}} \\
\hline
PF-CHAT-001 & Envío de Mensaje & 3 & 1 & 3 & 2 & 9 & Chatbot \\
PF-CHAT-002 & Historial de Chat & 1 & 1 & 1 & 0 & 3 & Chatbot \\
PF-DOC-001 & Carga de Documento & 2 & 1 & 0 & 2 & 5 & Chatbot \\
PF-DOC-002 & Procesamiento de Doc & 3 & 1 & 2 & 2 & 8 & Chatbot \\
PF-DOC-003 & Listar Documentos & 0 & 1 & 1 & 0 & 2 & Chatbot \\
PF-DOC-004 & Eliminar Documento & 1 & 1 & 1 & 2 & 5 & Chatbot \\
\hline
\multicolumn{2}{|l|}{\textbf{Subtotal Chatbot}} & 10 & 6 & 8 & 8 & \textbf{32} & \\
\hline
\hline
\multicolumn{8}{|l|}{\cellcolor{gray!30}\textbf{ORCHARD SERVICE}} \\
\hline
PF-ORC-001 & Crear Huerto & 5 & 1 & 0 & 1 & 7 & Orchard \\
PF-ORC-002 & Consulta por ID & 1 & 1 & 1 & 0 & 3 & Orchard \\
PF-ORC-003 & Listar Huertos & 1 & 1 & 1 & 0 & 3 & Orchard \\
PF-ORC-004 & Actualizar Huerto & 6 & 1 & 1 & 1 & 9 & Orchard \\
PF-ORC-005 & Eliminar Huerto & 1 & 1 & 1 & 1 & 4 & Orchard \\
PF-ORC-006 & Activar Huerto & 1 & 1 & 1 & 1 & 4 & Orchard \\
PF-ORC-007 & Desactivar Huerto & 1 & 1 & 1 & 1 & 4 & Orchard \\
PF-ORC-008 & Agregar Planta & 2 & 1 & 1 & 1 & 5 & Orchard \\
PF-ORC-009 & Remover Planta & 2 & 1 & 1 & 1 & 5 & Orchard \\
\hline
\multicolumn{2}{|l|}{\textbf{Subtotal Orchard}} & 20 & 9 & 9 & 6 & \textbf{44} & \\
\hline
\hline
\multicolumn{2}{|l|}{\cellcolor{blue!20}\textbf{TOTAL GENERAL}} & \textbf{51} & \textbf{23} & \textbf{25} & \textbf{18} & \textbf{117} & \\
\hline
\end{longtable}

\section{Cálculo Paso a Paso}

\subsection{Verificación de Totales}

\textbf{Suma de movimientos de datos:}

\begin{align*}
\text{Total Entry (E)} &= 6 + 15 + 10 + 20 = 51 \text{ CFPs} \\
\text{Total Exit (X)} &= 3 + 5 + 6 + 9 = 23 \text{ CFPs} \\
\text{Total Read (R)} &= 2 + 6 + 8 + 9 = 25 \text{ CFPs} \\
\text{Total Write (W)} &= 1 + 3 + 8 + 6 = 18 \text{ CFPs} \\
\hline
\text{\textbf{TOTAL CFPs}} &= 51 + 23 + 25 + 18 = \textbf{117 CFPs}
\end{align*}

\subsection{Distribución Porcentual}

\begin{center}
\begin{tabular}{|l|r|r|}
\hline
\textbf{Microservicio} & \textbf{CFPs} & \textbf{\% del Total} \\
\hline
Authentication Service & 12 & 10.3\% \\
Users Service & 29 & 24.8\% \\
Chatbot Service & 32 & 27.4\% \\
Orchard Service & 44 & 37.6\% \\
\hline
\textbf{Total} & \textbf{117} & \textbf{100.0\%} \\
\hline
\end{tabular}
\end{center}

\subsection{Distribución por Tipo de Movimiento}

\begin{center}
\begin{tabular}{|l|r|r|}
\hline
\textbf{Tipo de Movimiento} & \textbf{Cantidad} & \textbf{\% del Total} \\
\hline
Entry (E) & 51 & 43.6\% \\
Read (R) & 25 & 21.4\% \\
Exit (X) & 23 & 19.7\% \\
Write (W) & 18 & 15.4\% \\
\hline
\textbf{Total} & \textbf{117} & \textbf{100.0\%} \\
\hline
\end{tabular}
\end{center}

\textbf{Interpretación}: La distribución muestra un sistema orientado a la entrada de datos (43.6\% de movimientos Entry), característico de APIs REST que reciben múltiples parámetros. El balance entre Read (21.4\%) y Write (15.4\%) indica un sistema con operaciones tanto de consulta como de modificación.

\newpage

\section{Supuestos y Decisiones de Medición}

\subsection{Supuestos Generales}

\begin{enumerate}
    \item \textbf{SUP-001 - Límite del Software}: El servicio Ollama se considera fuera del límite del software como un usuario funcional externo que provee servicios de IA.

    \textit{Justificación}: Ollama es un servicio de terceros independiente, no forma parte del código del sistema medido. Las interacciones con Ollama se contabilizan como movimientos Read hacia un usuario funcional externo.

    \item \textbf{SUP-002 - Almacenamiento Persistente}: ChromaDB se considera almacenamiento persistente dentro del límite del software.

    \textit{Justificación}: ChromaDB es gestionado directamente por el microservicio de chatbot y sus datos son esenciales para la funcionalidad RAG. Los movimientos hacia ChromaDB son Write/Read estándar.

    \item \textbf{SUP-003 - API Gateway}: El API Gateway no cuenta como procesos funcionales adicionales, actúa como proxy transparente.

    \textit{Justificación}: Según COSMIC 4.0, los componentes de infraestructura que solo enrutan sin transformar datos no constituyen procesos funcionales propios. El gateway solo redirige peticiones HTTP.

    \item \textbf{SUP-004 - Servicios No Implementados}: Los servicios de notificaciones y algoritmo genético mencionados en las rutas no fueron contabilizados por falta de implementación visible.

    \textit{Justificación}: Solo se encuentran referencias en el router del API Gateway pero no hay controladores ni casos de uso implementados en el código analizado.

    \item \textbf{SUP-005 - Health Checks}: Los endpoints de health check no se contabilizan como procesos funcionales.

    \textit{Justificación}: Los health checks son funciones técnicas de infraestructura, no requisitos funcionales del usuario según COSMIC.
\end{enumerate}

\subsection{Decisiones de Contabilización}

\begin{enumerate}
    \item \textbf{DEC-001 - Parámetros Opcionales}: Los parámetros opcionales en endpoints PUT/PATCH se cuentan como Entry aunque no siempre se envíen.

    \textit{Justificación}: COSMIC mide la funcionalidad máxima disponible, no el uso promedio. Si el endpoint acepta el parámetro, se cuenta.

    \item \textbf{DEC-002 - Validaciones en Base de Datos}: Las consultas de validación (ej: verificar email duplicado) se cuentan como movimientos Read independientes.

    \textit{Justificación}: Cada acceso a almacenamiento persistente con propósito distinto constituye un movimiento independiente.

    \item \textbf{DEC-003 - Interacciones con Ollama}: Cada llamada al servicio Ollama (embeddings, chat completion) se cuenta como Read.

    \textit{Justificación}: Ollama es un usuario funcional externo que provee datos (embeddings, respuestas LLM) al proceso.

    \item \textbf{DEC-004 - Almacenamiento de Archivos}: El sistema de archivos se considera almacenamiento persistente para PDFs.

    \textit{Justificación}: Los archivos persisten más allá de la ejecución del proceso funcional y son recuperables posteriormente.
\end{enumerate}

\subsection{Observaciones Relevantes}

\begin{enumerate}
    \item \textbf{OBS-001 - Complejidad del Chatbot}: El proceso de envío de mensaje al chatbot (PF-CHAT-001) tiene alta complejidad funcional con 9 CFPs debido a la integración con RAG.

    \textit{Impacto}: Este proceso involucra múltiples interacciones con servicios externos (Ollama para embeddings y generación) y base de datos vectorial (ChromaDB para búsqueda semántica), además de persistencia de historial.

    \item \textbf{OBS-002 - Autenticación Delegada}: El servicio de autenticación delega el almacenamiento de usuarios al servicio Users.

    \textit{Impacto}: Esto genera movimientos de lectura/escritura adicionales entre microservicios que fueron contabilizados correctamente como interacciones con almacenamiento persistente (MongoDB vía Users Service).

    \item \textbf{OBS-003 - Gestión de Plantas en Huertos}: Las operaciones de agregar/remover plantas modifican arrays embebidos en el documento del huerto.

    \textit{Decisión}: Cada operación cuenta como Write independiente aunque modifiquen el mismo documento, ya que representan requisitos funcionales distintos.
\end{enumerate}

\subsection{Riesgos y Limitaciones}

\begin{enumerate}
    \item \textbf{RIE-001 - Posible Subdimensionamiento}: Si los servicios de notificaciones y algoritmo genético están parcialmente implementados, el tamaño real podría ser mayor.

    \textit{Mitigación}: Se recomienda revisión adicional cuando estos servicios sean completamente implementados y documentados.

    \item \textbf{RIE-002 - Dependencia de Servicios Externos}: El sistema depende fuertemente de Ollama para funcionalidad core del chatbot.

    \textit{Observación}: La no disponibilidad de Ollama afectaría significativamente la funcionalidad medida (32 CFPs del chatbot, 27.4\% del total).
\end{enumerate}

\newpage

\section{Evidencia y Trazabilidad}

\subsection{Archivos de Código Analizados}

La medición se basó en análisis estático de los siguientes archivos fuente:

\subsubsection{API Gateway}
\begin{itemize}
    \item \texttt{api-gateway/src/app.ts}
    \item \texttt{api-gateway/src/routes/index.ts}
    \item \texttt{api-gateway/src/services/proxy.ts}
    \item \texttt{api-gateway/src/middleware/validateTokenWithAuthService.ts}
\end{itemize}

\subsubsection{Authentication Service}
\begin{itemize}
    \item \texttt{authentication/src/presentation/controllers/AuthController.ts}
    \item \texttt{authentication/src/presentation/routes/AuthRoutes.ts}
    \item \texttt{authentication/src/application/usecases/RegisterUseCase.ts}
    \item \texttt{authentication/src/application/usecases/LoginUseCase.ts}
    \item \texttt{authentication/src/application/usecases/ValidateTokenUseCase.ts}
\end{itemize}

\subsubsection{Users Service}
\begin{itemize}
    \item \texttt{api-users/src/presentation/controllers/UserController.ts}
    \item \texttt{api-users/src/presentation/routes/UserRoutes.ts}
    \item \texttt{api-users/src/application/use-cases/CreateUserUseCase.ts}
    \item \texttt{api-users/src/application/use-cases/GetUserByIdUseCase.ts}
    \item \texttt{api-users/src/application/use-cases/GetUserByEmailUseCase.ts}
    \item \texttt{api-users/src/application/use-cases/UpdateUserByIdUseCase.ts}
    \item \texttt{api-users/src/application/use-cases/DeleteUserByIdUseCase.ts}
\end{itemize}

\subsubsection{Chatbot Service}
\begin{itemize}
    \item \texttt{api-chatbot/src/presentation/controllers/ChatController.ts}
    \item \texttt{api-chatbot/src/presentation/controllers/DocumentController.ts}
    \item \texttt{api-chatbot/src/presentation/routes/ChatRoutes.ts}
    \item \texttt{api-chatbot/src/presentation/routes/DocumentRoutes.ts}
    \item \texttt{api-chatbot/src/application/use-cases/SendMessageUseCase.ts}
    \item \texttt{api-chatbot/src/application/use-cases/GetChatHistoryUseCase.ts}
    \item \texttt{api-chatbot/src/application/use-cases/UploadDocumentUseCase.ts}
    \item \texttt{api-chatbot/src/application/use-cases/ProcessDocumentUseCase.ts}
    \item \texttt{api-chatbot/src/application/use-cases/GetDocumentsUseCase.ts}
    \item \texttt{api-chatbot/src/application/use-cases/DeleteDocumentUseCase.ts}
\end{itemize}

\subsubsection{Orchard Service}
\begin{itemize}
    \item \texttt{api-orchard/src/presentation/controllers/OrchardController.ts}
    \item \texttt{api-orchard/src/presentation/routes/OrchardRoutes.ts}
    \item \texttt{api-orchard/src/application/use-cases/CreateOrchardUseCase.ts}
    \item \texttt{api-orchard/src/application/use-cases/GetOrchardUseCase.ts}
    \item \texttt{api-orchard/src/application/use-cases/ListOrchardsUseCase.ts}
    \item \texttt{api-orchard/src/application/use-cases/UpdateOrchardUseCase.ts}
    \item \texttt{api-orchard/src/application/use-cases/DeleteOrchardUseCase.ts}
    \item \texttt{api-orchard/src/application/use-cases/ToggleOrchardStateUseCase.ts}
    \item \texttt{api-orchard/src/application/use-cases/ManagePlantsUseCase.ts}
\end{itemize}

\subsection{Metodología de Análisis}

\begin{enumerate}
    \item \textbf{Identificación de Endpoints}: Revisión de archivos de rutas y controladores
    \item \textbf{Mapeo a Requisitos Funcionales}: Análisis de la funcionalidad implementada en cada endpoint
    \item \textbf{Conteo de Movimientos Entry}: Análisis de parámetros en \texttt{req.body}, \texttt{req.params}, \texttt{req.query}
    \item \textbf{Conteo de Movimientos Exit}: Análisis de respuestas en \texttt{res.json()}
    \item \textbf{Conteo de Movimientos Read/Write}: Análisis de casos de uso y repositorios
    \item \textbf{Validación Cruzada}: Verificación de consistencia entre controladores y casos de uso
\end{enumerate}

\subsection{Documento de Referencia}

La estructura y metodología de este documento se basan en:

\textbf{Docs/PuntosFuncionCosmic.pdf} - Documento de referencia que contiene un ejemplo de medición COSMIC para el proyecto PIWEB, utilizado como guía metodológica y de formato.

\newpage

\section{Anexos}

\subsection{Anexo A: Glosario de Términos}

\begin{description}
    \item[CFP] COSMIC Function Point - Unidad de medida COSMIC
    \item[Entry (E)] Movimiento de datos desde usuario funcional hacia el proceso
    \item[Exit (X)] Movimiento de datos desde el proceso hacia usuario funcional
    \item[Read (R)] Movimiento de datos desde almacenamiento persistente al proceso
    \item[Write (W)] Movimiento de datos desde el proceso a almacenamiento persistente
    \item[RAG] Retrieval-Augmented Generation - Técnica de IA que combina búsqueda semántica con generación de lenguaje
    \item[LLM] Large Language Model - Modelo de lenguaje de gran escala
    \item[Embedding] Representación vectorial de texto en espacio multidimensional
    \item[ChromaDB] Base de datos vectorial para búsqueda semántica
    \item[Ollama] Servicio local de modelos de lenguaje de código abierto
\end{description}

\subsection{Anexo B: Referencias Normativas}

\begin{enumerate}
    \item COSMIC Measurement Manual v4.0.2, The COSMIC Functional Size Measurement Method, Version 4.0.2, January 2017
    \item ISO/IEC 19761:2011 - Software engineering - COSMIC: a functional size measurement method
    \item COSMIC Implementation Guide for REST APIs (aplicable a arquitecturas de microservicios)
\end{enumerate}

\subsection{Anexo C: Información del Proyecto}

\begin{description}
    \item[Nombre del Proyecto] ApiGateway - Sistema de Microservicios
    \item[Tecnologías] Node.js, TypeScript, Express, MongoDB, ChromaDB, Ollama
    \item[Arquitectura] Microservicios con API Gateway
    \item[Patrón Arquitectónico] Clean Architecture / Hexagonal
    \item[Fecha de Análisis] \today
    \item[Versión Analizada] Versión actual del repositorio
\end{description}

\subsection{Anexo D: Contacto}

Para consultas sobre esta medición:

\begin{description}
    \item[Analista] Especialista Certificado COSMIC (CCFL)
    \item[Metodología] COSMIC v4.0
    \item[Archivos Generados]
    \begin{itemize}
        \item \texttt{PuntosFuncion\_COSMIC.tex} (este documento)
        \item \texttt{puntos\_cosmic.json} (datos estructurados)
    \end{itemize}
\end{description}

\newpage

\section{Conclusiones}

\subsection{Resumen de Hallazgos}

El sistema ApiGateway presenta las siguientes características de tamaño funcional:

\begin{itemize}
    \item \textbf{Tamaño Total}: 117 CFPs
    \item \textbf{Número de Procesos Funcionales}: 23
    \item \textbf{Promedio por Proceso}: 5.09 CFPs/proceso
    \item \textbf{Servicio Más Complejo}: Orchard Service (44 CFPs, 37.6\%)
    \item \textbf{Servicio Menos Complejo}: Authentication Service (12 CFPs, 10.3\%)
\end{itemize}

\subsection{Interpretación del Tamaño}

Con 117 CFPs, este sistema se clasifica como:

\begin{itemize}
    \item \textbf{Categoría}: Sistema de mediana complejidad funcional
    \item \textbf{Complejidad Distribuida}: Arquitectura de microservicios con responsabilidades bien definidas
    \item \textbf{Funcionalidad Especializada}: 27.4\% del tamaño corresponde a funcionalidades avanzadas de IA (Chatbot con RAG)
\end{itemize}

\subsection{Aplicaciones de esta Medición}

Los 117 CFPs medidos pueden utilizarse para:

\begin{enumerate}
    \item \textbf{Estimación de Esfuerzo}: Aplicando productividad histórica (CFP/hora)
    \item \textbf{Estimación de Costos}: Multiplicando por costo unitario por CFP
    \item \textbf{Benchmarking}: Comparación con proyectos similares de la industria
    \item \textbf{Control de Cambios}: Medición incremental de nuevas funcionalidades
    \item \textbf{Métricas de Productividad}: Líneas de código por CFP, defectos por CFP, etc.
\end{enumerate}

\subsection{Recomendaciones}

\begin{enumerate}
    \item Completar la implementación de los servicios de notificaciones y algoritmo genético para obtener una medición completa del sistema
    \item Considerar una re-medición cuando se agreguen funcionalidades significativas
    \item Utilizar esta línea base (117 CFPs) para proyecciones de crecimiento futuro
    \item Documentar la productividad real del equipo en CFP/hora para mejorar estimaciones futuras
\end{enumerate}

\vfill

\begin{center}
\rule{0.5\textwidth}{0.4pt}

\textbf{FIN DEL DOCUMENTO}

\vspace{0.5cm}

Documento generado por: Especialista Certificado COSMIC (CCFL)

Fecha: \today

\rule{0.5\textwidth}{0.4pt}
\end{center}

\end{document}
